\documentclass[a4paper,11pt]{book}
\usepackage{listings}
\usepackage[utf8]{inputenc}
\usepackage[spanish]{babel}

\decimalpoint
\usepackage{dcolumn}
\newcolumntype{.}{D{.}{\esperiod}{-1}}
\makeatletter
\addto\shorthandsspanish{\let\esperiod\es@period@code}
\makeatother

\RequirePackage{verbatim}
\usepackage{fancyhdr}
\usepackage{graphicx}
\usepackage{afterpage}
\usepackage{longtable}
\usepackage[pdfborder={000}]{hyperref}
\usepackage{url}
\usepackage{colortbl,longtable}
\usepackage[stable]{footmisc}
\usepackage{index}

% ********************************************************************
% Re-usable information
% ********************************************************************
\newcommand{\myTitle}{Lazarillo: plataforma robótica con Edge Computing para guía de caminos.\xspace}
\newcommand{\myDegree}{Máster en Ingeniería Informática\xspace}
\newcommand{\myName}{Adrián Morente Gabaldón\xspace}
\newcommand{\myProf}{Juan José Ramos Muñoz\xspace}
\newcommand{\myFaculty}{Escuela Técnica Superior de Ingenierías Informática y de Telecomunicación\xspace}
\newcommand{\myFacultyShort}{E.T.S. de Ingenierías Informática y de Telecomunicación\xspace}
\newcommand{\myDepartment}{Departamento de ...\xspace}
\newcommand{\myUni}{\protect{Universidad de Granada}\xspace}
\newcommand{\myLocation}{Granada\xspace}
\newcommand{\myTime}{\today\xspace}
\newcommand{\myVersion}{Versión 1.0\xspace}

\hypersetup{
pdfauthor = {\myName (adrian95morente@gmail.com)},
pdftitle = {\myTitle},
pdfsubject = {},
pdfkeywords = {},
pdfcreator = {},
pdfproducer = {pdflatex}
}

%\makeindex
%\usepackage[style=long, cols=2,border=plain,toc=true,number=none]{glossary}
% \makeglossary

% Definición de comandos que me son tiles:
%\renewcommand{\indexname}{Índice alfabético}
%\renewcommand{\glossaryname}{Glosario}

\pagestyle{fancy}
\fancyhf{}
\fancyhead[LO]{\leftmark}
\fancyhead[RE]{\rightmark}
\fancyhead[RO,LE]{\textbf{\thepage}}
\renewcommand{\chaptermark}[1]{\markboth{\textbf{#1}}{}}
\renewcommand{\sectionmark}[1]{\markright{\textbf{\thesection. #1}}}

\setlength{\headheight}{1.5\headheight}

\newcommand{\HRule}{\rule{\linewidth}{0.5mm}}
%Definimos los tipos teorema, ejemplo y definición podremos usar estos tipos
%simplemente poniendo \begin{teorema} \end{teorema} ...
\newtheorem{teorema}{Teorema}[chapter]
\newtheorem{ejemplo}{Ejemplo}[chapter]
\newtheorem{definicion}{Definición}[chapter]

\definecolor{gray97}{gray}{.97}
\definecolor{gray75}{gray}{.75}
\definecolor{gray45}{gray}{.45}
\definecolor{gray30}{gray}{.94}

\lstset{ frame=Ltb,
     framerule=0.5pt,
     aboveskip=0.5cm,
     framextopmargin=3pt,
     framexbottommargin=3pt,
     framexleftmargin=0.1cm,
     framesep=0pt,
     rulesep=.4pt,
     backgroundcolor=\color{gray97},
     rulesepcolor=\color{black},
     %
     stringstyle=\ttfamily,
     showstringspaces = false,
     basicstyle=\scriptsize\ttfamily,
     commentstyle=\color{gray45},
     keywordstyle=\bfseries,
     %
     numbers=left,
     numbersep=6pt,
     numberstyle=\tiny,
     numberfirstline = false,
     breaklines=true,
   }
 
% minimizar fragmentado de listados
\lstnewenvironment{listing}[1][]
   {\lstset{#1}\pagebreak[0]}{\pagebreak[0]}

\lstdefinestyle{CodigoC}
   {
	basicstyle=\scriptsize,
	frame=single,
	language=C,
	numbers=left
   }
\lstdefinestyle{CodigoC++}
   {
	basicstyle=\small,
	frame=single,
	backgroundcolor=\color{gray30},
	language=C++,
	numbers=left
   }
\lstdefinestyle{Consola}
   {basicstyle=\scriptsize\bf\ttfamily,
    backgroundcolor=\color{gray30},
    frame=single,
    numbers=none
   }
\newcommand{\bigrule}{\titlerule[0.5mm]}

%Para conseguir que en las páginas en blanco no ponga cabecerass
\makeatletter
\def\clearpage{%
  \ifvmode
    \ifnum \@dbltopnum =\m@ne
      \ifdim \pagetotal <\topskip
        \hbox{}
      \fi
    \fi
  \fi
  \newpage
  \thispagestyle{empty}
  \write\m@ne{}
  \vbox{}
  \penalty -\@Mi
}
\makeatother

\usepackage{pdfpages}
\begin{document}
\begin{titlepage}

\newlength{\centeroffset}
\setlength{\centeroffset}{-0.5\oddsidemargin}
\addtolength{\centeroffset}{0.5\evensidemargin}
\thispagestyle{empty}

\noindent\hspace*{\centeroffset}\begin{minipage}{\textwidth}

\centering
\includegraphics[width=0.9\textwidth]{imagenes/logo_ugr.jpg}\\[1.4cm]

\textsc{\Large TRABAJO FIN DE MÁSTER\\[0.2cm]}
\textsc{MÁSTER EN INGENIERÍA INFORMÁTICA}\\[1cm]
% Upper part of the page
% 
% Title
{\Huge\bfseries Lazarillo\\
}
\noindent\rule[-1ex]{\textwidth}{3pt}\\[3.5ex]
{\large\bfseries Plataforma robótica con Edge Computing para guía de caminos}
\end{minipage}

\vspace{2.5cm}
\noindent\hspace*{\centeroffset}\begin{minipage}{\textwidth}
\centering

\textbf{Autor}\\ {Adrián Morente Gabaldón}\\[2.5ex]
\textbf{Director}\\
{Juan José Ramos Muñoz}\\[2cm]
\includegraphics[width=0.3\textwidth]{imagenes/etsiit_logo.png}\\[0.1cm]
\textsc{Escuela Técnica Superior de Ingenierías Informática y de Telecomunicación}\\
\textsc{---}\\
Granada, septiembre de 2022
\end{minipage}
%\addtolength{\textwidth}{\centeroffset}
%\vspace{\stretch{2}}
\end{titlepage}

\chapter*{}

%\cleardoublepage
\thispagestyle{empty}

\begin{center}
{\large\bfseries Lazarillo - Robot guía: Plataforma robótica de código abierto para uso general}\\
\end{center}
\begin{center}
Adrián Morente Gabaldón\\
\end{center}

%\vspace{0.7cm}
\noindent{\textbf{Palabras clave}: robot, embebido, \textit{IoT}, \textit{Linux}, \textit{Yocto}, web, \textit{C++}, \textit{PubSub}, \textit{React}, \textit{Docker}, \textit{Python}, \textit{Redis}}\\

\vspace{0.7cm}
\noindent{\textbf{Resumen}}\\

Lazarillo se trata de una plataforma libre de código abierto pensada para provisionar un robot que actúa como guía de caminos para los visitantes que acuden a la ETSIIT.\\

Este dispositivo embebido consta de un sistema operativo hecho a medida y una pantalla táctil para la interacción del usuario, además de diversas aplicaciones para su uso, e internamente una arquitectura software que permite extender sus funcionalidades a todos los desarrolladores interesados.\\

En cuanto a \textit{IoT}, la plataforma consta de métodos de conectividad que permiten al dispositivo ser administrado por un técnico desde un portal web, pudiendo así realizar actualizaciones o gestiones varias.
\cleardoublepage


\thispagestyle{empty}


\begin{center}
{\large\bfseries Lazarillo - Robot guide: Open-source multipurpose robotic platform}\\
\end{center}
\begin{center}
Adrián Morente Gabaldón\\
\end{center}

%\vspace{0.7cm}
\noindent{\textbf{Keywords}: robot, embedded, \textit{IoT}, \textit{Linux}, \textit{Yocto}, web, \textit{C++}, \textit{PubSub}, \textit{React}, \textit{Docker}, \textit{Python}, \textit{Redis}}\\

\vspace{0.7cm}
\noindent{\textbf{Abstract}}\\

Lazarillo is a free \& open-source platform for provisioning a robot that shall behave as a path guide for the ETSIIT's visitors.\\

This embedded device contains a custom-made operating system and a touchscreen that the user interacts with, in addition to some assorted applications. Internally, it contains a software architecture that allows any interested developers to extend its functionalities.\\

Regarding \textit{IoT}, the platform includes connectivity methods that make a technician able to manage or upgrade the device through a web portal.

\chapter*{}
\thispagestyle{empty}

\noindent\rule[-1ex]{\textwidth}{2pt}\\[4.5ex]

Yo, \textbf{Adrián Morente Gabaldón}, alumno de la titulación Máster en Ingeniería Informática de la \textbf{Escuela Técnica Superior de Ingenierías Informática y de Telecomunicación de la Universidad de Granada}, con DNI 77139229N, autorizo la ubicación de la siguiente copia de mi Trabajo Fin de Grado en la biblioteca del centro para que pueda ser consultada por las personas que lo deseen.

\vspace{6cm}

\noindent Fdo: Adrián Morente Gabaldón

\vspace{2cm}

\begin{flushright}
Granada a 9 de septiembre de 2022.
\end{flushright}

\chapter*{Agradecimientos}
\thispagestyle{empty}

       \vspace{1cm}


A Mari Carmen, Miguel y Lorena por animarme a cerrar esta etapa después de años posponiéndolo prestando atención a cosas más prioritarias. Porque ``el tiempo pone cada cosa en su lugar''.\\

A mis ex-compis de uni por estar siempre ahí para apoyarnos en los caminos tan diversos que vamos tomando.\\

A Paola, que pese a no soportar la frialdad y perfeccionismo con que me tomo las cosas, siempre me obliga a que termine lo que me da pereza hacer.


%\frontmatter
%\tableofcontents
%\listoffigures
%\listoftables
%
%\mainmatter
%\setlength{\parskip}{5pt}

\chapter{Introducción}

\textbf{Lazarillo} se trata de una plataforma robótica abierta cuyo propósito es proporcionar una arquitectura solvente y extensible que permita añadir nuevas características y utilidades, además de facilitar el acceso a su gestión y mantenimiento.\\

Contiene un sistema operativo abierto basado en GNU/Linux y \textit{The Yocto Project}, además de distintos servicios implementados con lenguajes de programación diferentes, mostrando así la interoperabilidad del software. Goza de conectividad inalámbrica, la cual facilita la conexión con herramientas externas para su gestión. Asímismo, el proyecto también consta de un panel web de administración desde el cual se pueden enviar acciones remotas al robot.\\

Aunque se trata de una plataforma extensible y de propósito múltiple, el uso inicial para el que fue ideado es el de actuar como \textbf{asistente} y \textbf{guía} dentro de un espacio cerrado (ya podemos ver que el título asignado al proyecto es un pequeño guiño a la literatura española). Sin embargo, el \textit{stack} de herramientas y arquitectura que se han ido confeccionando durante su desarrollo, se podrían utilizar fácilmente para cualquier proyecto de propósito general que aúne dispositivos embebidos con \textit{IoT} y administración remota de éstos.
\chapter{Especificación y requisitos}

Pasemos ahora a describir de la forma más detallada posible cada uno de los requerimientos que conforman la idea del proyecto, planteando del mismo modo alternativas o aspectos que sería de agrado incluir, si bien no forman parte de la especificación inicialmente. Las decisiones tomadas, así como las soluciones implementadas, serán detalladas en el capítulo siguiente, si bien a lo largo de éste mismo pueden surgir necesidades cuya solución se detalle directamente.\\

Se desea disponer de una plataforma robótica extensible, libre y abierta; que permita una buena ampliación de nuevas características mediante software. Además, se propone la implementación de un uso concreto para esta plataforma, y es que dicho robot sirva como \textbf{guía de caminos} para sus usuarios finales en un \textbf{entorno controlado}. En este capítulo listaremos los distintos requisitos y puntualizaremos sobre cada una de las decisiones tomadas para satisfacerlos.\\

\section{Requisitos generales}

Los requerimientos aquí listados comprenderán cosas tanto de procedimientos para el desempeño del proyecto (como pueden ser la visibilidad y su legislación) hasta las funcionalidades más concretas que se esperan del producto final.\\

\subsection{Licencias}

Se tratará de un proyecto de software \textbf{libre} y de \textbf{código abierto}. Para ello, se publicará bajo una licencia \textit{GPLv3} en un repositorio público en \textit{Github}.\\

Además de \textit{Github}, existen otras alternativas de repositorios públicos que permiten utilizar \textit{git} para control de versiones (como \textit{Bitbucket} y \textit{Gitlab} entre otras). El porqué de utilizar \textit{Github} es meramente por aprovechar la licencia \textit{premium} que se provee a los estudiantes de la UGR simplemente por matricularse; permitiendo así tener algunos repositorios privados a su disposición \cite{github-premium}.\\

Con el código público y la licencia elegida, cualquier usuario podrá descargar el software, compilarlo, ejecutarlo e incluso añadir modificaciones al código para ser probadas e integradas en la plataforma final. Para ello, en el propio repositorio se pondrá a disposición de los interesados una documentación que explique cómo replicar el entorno tanto de desarrollo como de compilación.\\


\subsection{Especificaciones}

El término \textit{plataforma robótica} puede ser demasiado amplio para su manejo, por lo que en esta sección detallaremos más a fondo algunos de los puntos más interesantes, así como los factores de éxito que harían de \textbf{\textit{Lazarillo}} un producto útil y diferencial con respecto a las alternativas ya existentes.\\


\subsubsection{Extensibilidad}

Ya que se desea disponer de una plataforma extensible cuyo comportamiento y características puedan ampliarse a través de software, se debe dotar al robot de una arquitectura que permita este crecimiento, conteniendo en ella servicios (o módulos) independientes que puedan incluirse o no en función de la aplicación específica que vaya a cumplir el robot.\\

Para ello, sería interesante disponer de una arquitectura basada en \textbf{microservicios} donde cada uno de ellos cumple un propósito muy concreto y se comunica con el resto sin generar acoplamiento. Para esto es de imperativa necesidad utilizar un paradigma de comunicación en el cual sea transparente añadir datos y servicios.\\

Por otro lado, ha de asegurarse que existe la \textbf{interoperabilidad}, la cual representa que una arquitectura que incluye distintas plataformas de hardware y diferentes sistemas software (implementados con lenguajes de programación variados) cooperan bien entre sí.\\

Veamos un ejemplo rápido para ilustrar el párrafo anterior: si el programa que recibe los datos de un servidor web está programado en \textit{Python} y debe transmitirlos al servicio que toma las decisiones de movimiento (programado en \textit{C++}); el paradigma de comunicación que los conecta ha de ser \textbf{agnóstico en el lenguaje} y que dicha operación sea efectiva de forma transparente.\\


\subsubsection{Conectividad}

Como plataforma inteligente y conectada que utiliza el paradigma del \textbf{\textit{edge computing}}, sabemos que el robot ha de ser un dispositivo embebido con conexiones al exterior como \textit{Bluetooth} y/o \textit{WiFi}. Que éstas vengan implícitas en la plataforma hardware utilizada facilitará mucho el trabajo, ya que nos ahorramos el hecho de tener que soldar componentes. En cuanto a plataformas hardware, en el apartado del \textbf{Estado del Arte} ya comentamos algunas alternativas y opciones. Utilizaremos para ésto una \textit{Raspberry Pi Model 3 B}, que pese a no ser el último modelo de la marca \textit{Raspberry}, es la que tengo a disposición en casa. Además, satisface las necesidades de conectividad que comentábamos.\\

Por otro lado, en cuanto al requisito de computación en el borde, la plataforma deberá contener uno o más servicios que permitan la comunicación con el exterior, de una forma u otra, además de enviar y recibir mensajes. El robot deberá habilitar un canal de comunicación \textbf{persistente} y \textbf{bidireccional} que le permitan tanto a él como al servidor enviarse eventos entre sí.\\


\subsubsection{Interfaz y experiencia de usuario}

El robot contará con una pantalla táctil con la que proveerá la información necesaria al usuario (en función de las aplicaciones que necesite o decidan integrarse en el robot). Cualquier pantalla táctil que permita su conexión con la \textit{Raspberry} debería servir, por lo que no entraremos a detallar limitaciones hardware. Para la interacción del usuario, el sistema contará con una \textbf{aplicación embebida de entorno gráfico} que permita el uso del robot.\\

Dado que \textbf{\textit{Lazarillo}} se pretende que actúe como \textbf{guía}, otra característica que sería de agradecer en la plataforma tiene que ver con la \textbf{reproducción de sonidos} que faciliten la comunicación con el usuario, así como la \textbf{accesibilidad}. No todo el mundo goza de capacidad visual o simplemente no están habituados a interfaces táctiles, por lo que emitir alertas y sonidos descriptivos facilitaría llegar a más usuarios de forma plena y satisfactoria.\\

\subsubsection{Gestión experta y mantenimiento}

Como hemos comentado en secciones anteriores, el robot gozará de hardware provisto de conectividad inalámbrica. En este punto haremos uso de esta característica para ofrecer un método de mantenimiento, supervisión y gestión del robot, por parte de alguna "mano experta". Necesitaremos un método de administración de los distintos robots existentes desde un portal web externo a ellos. Un técnico encargado de gestionar los robots, accederá a una web alojada en un servidor a través del protocolo común de \textit{HTTP}.\\

Inicialmente, este portal web servirá para listar los dispositivos conectados (es decir, los robots que han sido provisionados con el software de \textit{Lazarillo} y se encuentran en funcionamiento), pero posteriormente permitirá enviar acciones remotas desde el servidor al robot (como reinicios, actualizaciones de software, acciones concretas a realizar por el robot, etc.).\\

Es deseable que la interfaz web sea sencilla y usable. Además, sería de agradecer que ésta pueda visualizarse correctamente en \textbf{dispositivos móviles}, ya que ampliaría el rango de posibilidades de gestión de los dispositivos robóticos.\\

\subsubsection{Movilidad}

El factor determinante que diferenciará a nuestro robot de un sistema empotrado inmóvil será la capacidad de desplazarse por el entorno. Ya sea para un uso u otro, el robot deberá venir dotado de un sistema hardware que le permita avanzar por el plano, conteniendo elementos como \textbf{motores} y \textbf{ruedas} o \textbf{cintas móviles}.\\

Además, si se desea que el robot sea \textbf{inteligente} y reconozca el entorno por el que se está moviendo, deberá dotarse de algún sistema de reconocimiento como \textbf{sensores de proximidad} o \textbf{cámaras}. Respecto a esto, si queremos seguir el paradigma de \textit{edge computing} como venimos comentando, el procesamiento de estas señales podría realizarse en el servidor en lugar de en el propio robot, lo que también liberaría a la plataforma hardware del robot de una buena parte de la computación.\\

Para acotar el alcance del proyecto y que sea asumible para un trabajo de este calibre, inicialmente la movilidad podrá estar basada en hacer \textbf{seguimiento de líneas} sobre el suelo.\\

\begin{figure}[h]
	\centering
	\includegraphics[width=0.6\textwidth]{imagenes/robotnik.jpg}
	\caption{Ejemplo de robot móvil autónomo - Fuente: \textit{Robotnik}}
\end{figure}


\section{Objetivos opcionales}

En esta sección enumeraremos y describiremos sucíntamente qué otras ideas surgieron durante el momento de \textit{brainstorming} del proyecto, y que si bien son opcionales para su desempeño, realmente aportarían algún valor al producto final.\\


\subsubsection{Movilidad autónoma}

Un factor que haría de \textbf{\textit{Lazarillo}} un producto totalmente independiente y útil sería que no necesitase de caminos guiados para desplazarse. Se valoraría la instalación en sí mismo de los mapas cerrados en que se ubicaría durante su desempeño, así como un mecanismo de \textbf{geolocalización} en el espacio. Contando con esto, el robot mantendría una comunicación persistente con el servidor informando en \textit{tiempo real} de la ubicación actual.\\






\chapter{Conclusiones y trabajos futuros}

Un buen \textit{brainstorming} sobre el desarrollo de una idea novedosa y (se supone que) útil siempre es motivador y consigue animar a un programador a llevar la idea a la práctica, intentando la consecución del mayor número de requisitos posibles. Sin embargo, el tiempo siempre es uno de los factores más determinantes a la hora de dictaminar cómo de lejos se llega con el proyecto, por lo que siempre existe la posibilidad de que no se alcance a finalizar todas las ideas que rondan la cabeza.\\

Personalmente, tener que compaginar mi jornada completa laboral con el desempeño de este trabajo ha hecho mucha mella en lo que fue la primera idea del proyecto. Me habría gustado poder acercarme realmente más al hardware, disponer de un robot sencillito al que implementar unas directivas muy básicas de movimiento pero que realmente hiciesen tangible el trabajo desarrollado en ésto.\\

También me gustaría destacar que, después de años de experiencia profesional, realmente quería tomar buenas decisiones, plantear bien la arquitectura, usar las herramientas idóneas para cada caso, establecer buenas prácticas a largo plazo para asegurar que si yo no puedo hacerlo, alguien podría acceder a los repositorios y darle continuidad a estas finalidades que han quedado en el limbo.\\

Cabe destacar que dada la naturaleza libre del proyecto, cualquier interesado o interesada podría continuar con la parte del desarrollo que más le interese; ya que esta idea nació como una herramienta abierta a la que poder contribuir y de la que cualquiera pueda sacar utilidad.\\

Dicho esto, en este capítulo comentaremos algunas de las ideas iniciales que no llegaron a finalizarse a tiempo para la entrega y cuáles se decidieron postergar en el tiempo para un futuro desarrollo.\\

Para empezar, lo más ligado a lo que pretendía ser el epicentro del proyecto: la \textbf{robótica}. Sería idóneo instalar el sistema operativo preparado al efecto en un robot real cuyo núcleo de computación sea la \textit{Raspberry}. Tras esto, conectar algunos motores y unas ruedas para realizar un movimiento autónomo aunque fuese muy sencillo (implementando el módulo \textbf{\textit{motor-manager}} que se comentó en la arquitectura).\\

En cuanto al resto del código embebido en el cargador, queda pendiente la aplicación embebida mostrada en la pantalla táctil. En ella el usuario introducirá el destino al que quiere ir, el despacho del que desea saber cómo llegar, etc.\\

En cuanto a los servicios web, un aspecto importante que no llegó a gestionarse fue el de la autenticación del administrador. Obviamente para una gestión experta de dispositivos delicados, no cualquiera debería tener acceso a las herramientas, por lo que los desarrolladores del servicio web deberían crear cuentas autorizadas para que los administradores puedan autenticarse en el portal.\\

Hilando un poco más fino, los archivos \textit{Dockerfile} con los que se generan los contenedores incluyen variables de entorno en texto plano como usuarios y contraseñas de la base de datos. Lógicamente esto para un proyecto de prueba es aceptable temporalmente pero orientado a desplegarse en producción, lo ideal es leer los valores de variables del sistema local pero que no aparezcan en ningún fichero.\\

En cuanto a la conectividad, ya lo hemos ido viendo, pero el socket que conecta la web con el robot debería hacer comprobaciones periódicas de la lista de dispositivos existentes.\\


%%\nocite{*}
%\bibliography{bibliografia/bibliografia}\addcontentsline{toc}{chapter}{Bibliografía}
%\bibliographystyle{miunsrturl}
%
%\appendix
%\input{apendices/manual_usuario/manual_usuario}
%%\input{apendices/paper/paper}
%\input{glosario/entradas_glosario}
% \addcontentsline{toc}{chapter}{Glosario}
% \printglossary
\chapter*{}
\thispagestyle{empty}

\end{document}
