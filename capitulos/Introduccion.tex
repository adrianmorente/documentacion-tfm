\chapter*{Introducción}

\textbf{Lazarillo} se trata de una plataforma robótica abierta cuyo propósito es proporcionar una arquitectura solvente y extensible que permita añadir nuevas características y utilidades.\\

Contiene un sistema operativo abierto basado en GNU/Linux y \textit{The Yocto Project}, además de distintos servicios implementados con lenguajes de programación diferentes, mostrando así la interoperabilidad del software. Goza de conectividad inalámbrica, se conecta con un servidor que hace las veces de centro computacional en un paradigma de \textit{edge computing} y está dotado de un panel web de administración (desde el cual se pueden enviar acciones remotas al robot).\\

Aunque se trata de una plataforma extensible y de propósito múltiple, el uso inicial para el que fue ideado es el de actuar como \textbf{asistente} y \textbf{guía} dentro de un espacio cerrado (ya podemos ver que el título asignado al proyecto es un pequeño guiño a la literatura española).
