\chapter*{}

%\cleardoublepage
\thispagestyle{empty}

\begin{center}
{\large\bfseries Lazarillo - Robot guía: Plataforma robótica de código abierto para uso general}\\
\end{center}
\begin{center}
Adrián Morente Gabaldón\\
\end{center}

%\vspace{0.7cm}
\noindent{\textbf{Palabras clave}: robot, embebido, \textit{IoT}, \textit{Linux}, web, C++}\\

\vspace{0.7cm}
\noindent{\textbf{Resumen}}\\

Lazarillo se trata de una plataforma libre de código abierto pensada para provisionar un robot que actúa como guía de caminos para los visitantes que acuden a la ETSIIT.\\

Este dispositivo embebido consta de un sistema operativo hecho a medida además de diversas aplicaciones para su uso, e internamente una arquitectura software que permite extender sus funcionalidades a todos los desarrolladores interesados.\\

En cuanto a \textit{IoT}, la plataforma consta de métodos de conectividad que permiten al dispositivo ser administrado por un técnico desde un portal web, pudiendo así realizar actualizaciones o gestiones varias.
\cleardoublepage


\thispagestyle{empty}


\begin{center}
{\large\bfseries Lazarillo - Robot guide: Open-source multipurpose robotic platform}\\
\end{center}
\begin{center}
Adrián Morente Gabaldón\\
\end{center}

%\vspace{0.7cm}
\noindent{\textbf{Keywords}: robot, embedded, \textit{IoT}, \textit{Linux}, web, C++}\\

\vspace{0.7cm}
\noindent{\textbf{Abstract}}\\

Lazarillo is a free \& open-source platform for provisioning a robot that shall behave as a path guide for the ETSIIT's visitors.\\

This embedded device contains a custom-made operating system in addition to some assorted applications. Internally, it contains a software architecture that allows any interested developers to extend its functionalities.\\

Regarding \textit{IoT}, the platform includes connectivity methods that make a technician able to manage or upgrade the device through a web portal.

\chapter*{}
\thispagestyle{empty}

\noindent\rule[-1ex]{\textwidth}{2pt}\\[4.5ex]

Yo, \textbf{Adrián Morente Gabaldón}, alumno de la titulación Máster en Ingeniería Informática de la \textbf{Escuela Técnica Superior de Ingenierías Informática y de Telecomunicación de la Universidad de Granada}, con DNI 77139229N, autorizo la ubicación de la siguiente copia de mi Trabajo Fin de Grado en la biblioteca del centro para que pueda ser consultada por las personas que lo deseen.

\vspace{6cm}

\noindent Fdo: Adrián Morente Gabaldón

\vspace{2cm}

\begin{flushright}
Granada a 9 de septiembre de 2022.
\end{flushright}

\chapter*{Agradecimientos}
\thispagestyle{empty}

       \vspace{1cm}


Poner aquí agradecimientos...

