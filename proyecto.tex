\documentclass[a4paper,10pt]{book}
\usepackage{listings}
\usepackage[utf8]{inputenc}
\usepackage[spanish]{babel}

\decimalpoint
\usepackage{dcolumn}
\newcolumntype{.}{D{.}{\esperiod}{-1}}
\makeatletter
\addto\shorthandsspanish{\let\esperiod\es@period@code}
\makeatother

\RequirePackage{verbatim}
\usepackage{fancyhdr}
\usepackage{graphicx}
\usepackage{afterpage}
\usepackage{longtable}
\usepackage{hyperref}
\usepackage{url}
\usepackage{colortbl,longtable}
\usepackage[stable]{footmisc}
\usepackage{index}
\usepackage{csquotes}

% Customizar colores de enlaces
\hypersetup{
	colorlinks=true,
	linkcolor=black,
	filecolor=magenta,
	urlcolor=cyan,
	pdfpagemode=FullScreen
}

% ********************************************************************
% Re-usable information
% ********************************************************************
\newcommand{\myTitle}{Lazarillo - Robot guía: Plataforma robótica de código abierto para uso general.\xspace}
\newcommand{\myDegree}{Máster en Ingeniería Informática\xspace}
\newcommand{\myName}{Adrián Morente Gabaldón\xspace}
\newcommand{\myProf}{Juan José Ramos Muñoz\xspace}
\newcommand{\myFaculty}{Escuela Técnica Superior de Ingenierías Informática y de Telecomunicación\xspace}
\newcommand{\myFacultyShort}{E.T.S. de Ingenierías Informática y de Telecomunicación\xspace}
\newcommand{\myDepartment}{Departamento de ...\xspace}
\newcommand{\myUni}{\protect{Universidad de Granada}\xspace}
\newcommand{\myLocation}{Granada\xspace}
\newcommand{\myTime}{\today\xspace}
\newcommand{\myVersion}{Versión 1.0\xspace}

\hypersetup{
pdfauthor = {\myName (adrian95morente@gmail.com)},
pdftitle = {\myTitle},
pdfsubject = {},
pdfkeywords = {},
pdfcreator = {},
pdfproducer = {pdflatex}
}

%\makeindex
%\usepackage[style=long, cols=2,border=plain,toc=true,number=none]{glossary}
% \makeglossary

% Definición de comandos que me son tiles:
%\renewcommand{\indexname}{Índice alfabético}
%\renewcommand{\glossaryname}{Glosario}

\pagestyle{fancy}
\fancyhf{}
\fancyhead[LO]{\leftmark}
\fancyhead[RE]{\rightmark}
\fancyhead[RO,LE]{\textbf{\thepage}}
\renewcommand{\chaptermark}[1]{\markboth{\textbf{#1}}{}}
\renewcommand{\sectionmark}[1]{\markright{\textbf{\thesection. #1}}}

\setlength{\headheight}{1.5\headheight}

\newcommand{\HRule}{\rule{\linewidth}{0.5mm}}
%Definimos los tipos teorema, ejemplo y definición podremos usar estos tipos
%simplemente poniendo \begin{teorema} \end{teorema} ...
\newtheorem{teorema}{Teorema}[chapter]
\newtheorem{ejemplo}{Ejemplo}[chapter]
\newtheorem{definicion}{Definición}[chapter]

\definecolor{gray97}{gray}{.97}
\definecolor{gray75}{gray}{.75}
\definecolor{gray45}{gray}{.45}
\definecolor{gray30}{gray}{.94}

\lstset{ frame=Ltb,
     framerule=0.5pt,
     aboveskip=0.5cm,
     framextopmargin=3pt,
     framexbottommargin=3pt,
     framexleftmargin=0.1cm,
     framesep=0pt,
     rulesep=.4pt,
     backgroundcolor=\color{gray97},
     rulesepcolor=\color{black},
     %
     stringstyle=\ttfamily,
     showstringspaces = false,
     basicstyle=\scriptsize\ttfamily,
     commentstyle=\color{gray45},
     keywordstyle=\bfseries,
     %
     numbers=left,
     numbersep=6pt,
     numberstyle=\tiny,
     numberfirstline = false,
     breaklines=true,
   }
 
% minimizar fragmentado de listados
\lstnewenvironment{listing}[1][]
   {\lstset{#1}\pagebreak[0]}{\pagebreak[0]}

\lstdefinestyle{CodigoC}
   {
	basicstyle=\scriptsize,
	frame=single,
	language=C,
	numbers=left
   }
\lstdefinestyle{CodigoC++}
   {
	basicstyle=\small,
	frame=single,
	backgroundcolor=\color{gray30},
	language=C++,
	numbers=left
   }
\lstdefinestyle{Consola}
   {basicstyle=\scriptsize\bf\ttfamily,
    backgroundcolor=\color{gray30},
    frame=single,
    numbers=none
   }
\newcommand{\bigrule}{\titlerule[0.5mm]}

%Para conseguir que en las páginas en blanco no ponga cabeceras
\makeatletter
\def\clearpage{%
  \ifvmode
    \ifnum \@dbltopnum =\m@ne
      \ifdim \pagetotal <\topskip
        \hbox{}
      \fi
    \fi
  \fi
  \newpage
  \thispagestyle{empty}
  \write\m@ne{}
  \vbox{}
  \penalty -\@Mi
}
\makeatother

\usepackage{pdfpages}
\begin{document}
\begin{titlepage}

\newlength{\centeroffset}
\setlength{\centeroffset}{-0.5\oddsidemargin}
\addtolength{\centeroffset}{0.5\evensidemargin}
\thispagestyle{empty}

\noindent\hspace*{\centeroffset}\begin{minipage}{\textwidth}

\centering
\includegraphics[width=0.9\textwidth]{imagenes/logo_ugr.jpg}\\[1.4cm]

\textsc{\Large TRABAJO FIN DE MÁSTER\\[0.2cm]}
\textsc{MÁSTER EN INGENIERÍA INFORMÁTICA}\\[1cm]
% Upper part of the page
% 
% Title
{\Huge\bfseries Lazarillo\\
}
\noindent\rule[-1ex]{\textwidth}{3pt}\\[3.5ex]
{\large\bfseries Plataforma robótica con Edge Computing para guía de caminos}
\end{minipage}

\vspace{2.5cm}
\noindent\hspace*{\centeroffset}\begin{minipage}{\textwidth}
\centering

\textbf{Autor}\\ {Adrián Morente Gabaldón}\\[2.5ex]
\textbf{Director}\\
{Juan José Ramos Muñoz}\\[2cm]
\includegraphics[width=0.3\textwidth]{imagenes/etsiit_logo.png}\\[0.1cm]
\textsc{Escuela Técnica Superior de Ingenierías Informática y de Telecomunicación}\\
\textsc{---}\\
Granada, septiembre de 2022
\end{minipage}
%\addtolength{\textwidth}{\centeroffset}
%\vspace{\stretch{2}}
\end{titlepage}

\chapter*{}

%\cleardoublepage
\thispagestyle{empty}

\begin{center}
{\large\bfseries Lazarillo - Robot guía: Plataforma robótica de código abierto para uso general}\\
\end{center}
\begin{center}
Adrián Morente Gabaldón\\
\end{center}

%\vspace{0.7cm}
\noindent{\textbf{Palabras clave}: robot, embebido, \textit{IoT}, \textit{Linux}, \textit{Yocto}, web, \textit{C++}, \textit{PubSub}, \textit{React}, \textit{Docker}, \textit{Python}, \textit{Redis}}\\

\vspace{0.7cm}
\noindent{\textbf{Resumen}}\\

Lazarillo se trata de una plataforma libre de código abierto pensada para provisionar un robot que actúa como guía de caminos para los visitantes que acuden a la ETSIIT.\\

Este dispositivo embebido consta de un sistema operativo hecho a medida y una pantalla táctil para la interacción del usuario, además de diversas aplicaciones para su uso, e internamente una arquitectura software que permite extender sus funcionalidades a todos los desarrolladores interesados.\\

En cuanto a \textit{IoT}, la plataforma consta de métodos de conectividad que permiten al dispositivo ser administrado por un técnico desde un portal web, pudiendo así realizar actualizaciones o gestiones varias.
\cleardoublepage


\thispagestyle{empty}


\begin{center}
{\large\bfseries Lazarillo - Robot guide: Open-source multipurpose robotic platform}\\
\end{center}
\begin{center}
Adrián Morente Gabaldón\\
\end{center}

%\vspace{0.7cm}
\noindent{\textbf{Keywords}: robot, embedded, \textit{IoT}, \textit{Linux}, \textit{Yocto}, web, \textit{C++}, \textit{PubSub}, \textit{React}, \textit{Docker}, \textit{Python}, \textit{Redis}}\\

\vspace{0.7cm}
\noindent{\textbf{Abstract}}\\

Lazarillo is a free \& open-source platform for provisioning a robot that shall behave as a path guide for the ETSIIT's visitors.\\

This embedded device contains a custom-made operating system and a touchscreen that the user interacts with, in addition to some assorted applications. Internally, it contains a software architecture that allows any interested developers to extend its functionalities.\\

Regarding \textit{IoT}, the platform includes connectivity methods that make a technician able to manage or upgrade the device through a web portal.

\chapter*{}
\thispagestyle{empty}

\noindent\rule[-1ex]{\textwidth}{2pt}\\[4.5ex]

Yo, \textbf{Adrián Morente Gabaldón}, alumno de la titulación Máster en Ingeniería Informática de la \textbf{Escuela Técnica Superior de Ingenierías Informática y de Telecomunicación de la Universidad de Granada}, con DNI 77139229N, autorizo la ubicación de la siguiente copia de mi Trabajo Fin de Grado en la biblioteca del centro para que pueda ser consultada por las personas que lo deseen.

\vspace{6cm}

\noindent Fdo: Adrián Morente Gabaldón

\vspace{2cm}

\begin{flushright}
Granada a 9 de septiembre de 2022.
\end{flushright}

\chapter*{Agradecimientos}
\thispagestyle{empty}

       \vspace{1cm}


A Mari Carmen, Miguel y Lorena por animarme a cerrar esta etapa después de años posponiéndolo prestando atención a cosas más prioritarias. Porque ``el tiempo pone cada cosa en su lugar''.\\

A mis ex-compis de uni por estar siempre ahí para apoyarnos en los caminos tan diversos que vamos tomando.\\

A Paola, que pese a no soportar la frialdad y perfeccionismo con que me tomo las cosas, siempre me obliga a que termine lo que me da pereza hacer.


% \frontmatter
\tableofcontents
%\listoffigures
%\listoftables
% \mainmatter
%\setlength{\parskip}{5pt}

\chapter{Introducción}

\textbf{Lazarillo} se trata de una plataforma robótica abierta cuyo propósito es proporcionar una arquitectura solvente y extensible que permita añadir nuevas características y utilidades, además de facilitar el acceso a su gestión y mantenimiento.\\

Contiene un sistema operativo abierto basado en GNU/Linux y \textit{The Yocto Project}, además de distintos servicios implementados con lenguajes de programación diferentes, mostrando así la interoperabilidad del software. Goza de conectividad inalámbrica, la cual facilita la conexión con herramientas externas para su gestión. Asímismo, el proyecto también consta de un panel web de administración desde el cual se pueden enviar acciones remotas al robot.\\

Aunque se trata de una plataforma extensible y de propósito múltiple, el uso inicial para el que fue ideado es el de actuar como \textbf{asistente} y \textbf{guía} dentro de un espacio cerrado (ya podemos ver que el título asignado al proyecto es un pequeño guiño a la literatura española). Sin embargo, el \textit{stack} de herramientas y arquitectura que se han ido confeccionando durante su desarrollo, se podrían utilizar fácilmente para cualquier proyecto de propósito general que aúne dispositivos embebidos con \textit{IoT} y administración remota de éstos.
\chapter{Estado del arte}

Dado que a lo largo de este documento estaremos hablando sobre dispositivos embebidos, \textit{Internet de las cosas} y plataformas inteligentes (como robots), antes deberemos situarnos un poco en el contexto actual para revisar qué hay en el horizonte, teniendo en cuenta tanto software como hardware y limitaciones técnicas.\\

\subsubsection{¿Qué es un ``robot''?}

Comencemos analizando la definición de la palabra \textbf{\textit{robot}}, que según el \textit{IEEE} (\textit{Instituto de Ingenieros Eléctricos y Electrónicos}, asociación mundial de expertos dedicada a la normalización del desarrollo en diversas áreas técnicas \cite{ieee}), no \textit{es algo fácil de definir}, pero una buena aproximación sería ``una máquina autónoma capaz de percibir el entorno, realizando cálculos para la toma de decisiones y acciones que aplica en el mundo real'' \cite{whats_a_robot}.\\

Un ejemplo de estos es la \textit{Roomba} de la marca \textit{iRobot}, la conocida aspiradora robótica que recorre las habitaciones de la casa de forma totalmente autónoma realizando una limpieza (más o menos a fondo) de ésta \cite{roomba}. Los cálculos que realiza este dispositivo han de serle suficientes para recorrer la casa de forma eficiente, esquivando obstáculos y cubriendo la mayor superficie posible de la vivienda. Para este tipo de computaciones, un dispositivo robótico ha de valerse de sensores de proximidad (u otros), cámaras, etc.\\

\begin{figure}[h]
	\centering
	\includegraphics[width=0.7\textwidth]{imagenes/irobot-sensors.png}
	\caption{Ejemplo de sensores en una \textit{Roomba} - Fuente: \textit{ResearchGate} \cite{roomba-sensors-rg}}
\end{figure}

Estas medidas de detección y reacción ante el entorno son también usadas por elementos de inteligencia artificial que sin embargo no llegan a ser considerados como ``robots''. Podemos encontrar un ejemplo en el \textit{Autopilot}, que es como llama la compañía automovilística de vehículos eléctricos \textit{Tesla} a su sistema de conducción automática \cite{tesla-ai}. Los complejísimos algoritmos de \textbf{visión computacional} que tienen lugar en el vehículo a la hora de observar el entorno en busca de otros coches, peatones, señales y obstáculos; también pasan por un entrenamiento de modelos de inteligencia artificial intenso. La diferencia en el uso de estos, es que (al menos actualmente), aún no existe la conducción plenamente autónoma, y por seguridad aún se requiere que haya un conductor al volante. ¿Podríamos entonces establecer el límite entre lo que es un ``robot'' y lo que no en función de si es necesaria la interacción humana?\\

Responder a ésto es siempre complejo pero interesante, así que me gustaría concluir esa pregunta volviendo al artículo de ``What is a Robot?'' del \textit{IEEE} \cite{whats_a_robot} con la divertida respuesta de \textit{Joseph Engelberger}:

\begin{displayquote}
	``No sé cómo definir un robot, ¡pero sé reconocer uno cuando lo veo!''
\end{displayquote}

\subsubsection{¿Qué necesito para hacer un robot?}

Si quisiéramos crear un nuevo robot, son varios los campos de conocimiento que deberíamos tener en cuenta. Para empezar, la \textbf{electrónica} es algo esencial, ya que su base será la que permitirá controlar el movimiento del dispositivo a través de pulsos de corriente, además de dotarlo de los periféricos de detección y comunicación que hemos mencionado previamente. Por otro lado, el campo de la \textbf{informática} permitirá extender las funcionalidades de este robot dotándole de un sistema operativo y añadiendo aplicaciones sobre la capa abstracta tejida por la electrónica.\\

Si hablamos de hardware, podemos empezar por adquirir una plataforma móvil ya preparada para su uso. Lo primero que puede pensar cualquiera es tomar una como la \textit{Roomba}, cuya movilidad y detección de obstáculos ya está implementada, y sobre ella añadir los componentes y prestaciones que queramos. El problema en cuanto a ésto es que se trata de una \textbf{plataforma cerrada}. No disponer del código fuente hace que no podamos extender el robot como queramos, y así el fabricante se reserva los derechos de su uso.\\

Si no podemos asumir el presupuesto de una plataforma ya montada o simplemente queremos hacer pruebas y divertirnos con la experiencia, hoy en día es fácil y asequible acceder a componentes electrónicos con los que crear pequeños robots y artefactos caseros. Un ejemplo de \textit{núcleo} para ésto sería una placa de \textit{Arduino}, que por unos 20 o 30 euros contiene las conexiones para añadirle módulos de cámara, altavoces, servomotores y cualquier cosa que se nos ocurra \cite{arduino-store}.\\

\begin{figure}[h]
	\centering
	\includegraphics[width=0.9\textwidth]{imagenes/zowi-robot-componentes.jpg}
	\caption{\textit{Zowi}, ejemplo de robot basado en Arduino mas sensores - Fuente: \textit{Balara} \cite{zowi-balara}}
\end{figure}

Esta placa de desarrollo contiene un microcontrolador que permite ser programado en código C muy fácilmente, pero está limitado a eso. Si queremos utilizar un sistema que nos permita gestionar más cosas que la simple ejecución de un código, como un sistema operativo completo, deberíamos dar el salto a algo más parecido a un ordenador común, para lo que puede servirnos la conocida \textit{Raspberry Pi} \cite{raspberry-pi}. Existen otras alternativas de la misma gama provenientes de \textit{Asus} o de \textit{NXP} \cite{nxp-imx}, pero nos ceñiremos a la primera ya que está muy enfocada al sector educativo y al público joven y goza de una comunidad muy numerosa. Lo que todos estos dispositivos comparten son diversas conexiones a través de las cuales podemos añadir los elementos que mencionábamos previamente para nuestro proyecto de robot, además de permitirnos acceso a la configuración de su sistema operativo (o incluso podemos confeccionar uno a nuestra medida, como veremos en el siguiente apartado).\\

\begin{figure}[h]
	\centering
	\includegraphics[width=0.75\textwidth]{imagenes/asus-tinker-conexiones.png}
	\caption{\textit{Tinker board}, la alternativa de \textit{Asus} - Fuente: \textit{Asus.com} \cite{asus-tinker}}
\end{figure}


\subsubsection{Sistema operativo, el cerebro del robot}

Continuando con la \textit{Raspberry}, el ordenador monoplaca en torno al cual podríamos construir nuestro robot, pasemos ahora a hablar del sistema operativo. Para esta placa en cuestión existe uno muy aclamado por la gente llamado \textbf{\textit{Raspbian}} \cite{raspbian}, basado en \textit{Debian}, una conocida distribución de \textit{GNU/Linux} que viene de los propios creadores de la \textit{RPi} y dispone de todo lo necesario para formar un sistema de operativo completo, incluyendo hasta interfaz gráfica y aplicaciones de escritorio.\\

Contando con instalar en la tarjeta de memoria un sistema operativo ya montado como éste, podríamos seguir adelante con la conexión de periféricos que conformarán nuestro robot, o quizás nos interese más detenernos un momento a decidir si realmente queremos usar opciones como \textit{Raspbian} o preferimos tener algo personalizado y a nuestra medida.\\

Para los más aventureros, existe la posibilidad de diseñar y compilar un sistema operativo propio (aunque también basado en \textit{GNU/Linux}) con herramientas como \textbf{\textit{Buildroot}} \cite{buildroot} y \textbf{\textit{The Yocto Project}} \cite{yocto-project}. Se tratan de frameworks que permiten compilar una distribución \textit{Linux} a medida según las necesidades del dispositivo embebido en cuestión seleccionando las piezas que lo conforman como si de un puzzle se tratase. Por ejemplo, un robot que no tenga pantalla pero sí informe al usuario mediante efectos sonoros puede prescindir de todos los paquetes gráficos; igual que otro que no necesite conectarse a un router de forma inalámbrica no necesita tener instalados todos los paquetes correspondientes a la gestión de \textit{WiFi}.\\

Por un lado, tenemos la decisión entre utilizar un sistema operativo ya compilado o diseñarlo como hemos comentado. Si miramos el estudio que realizó \textit{Mads Doré Hansen} en su paper \textit{Yocto or Debian for Embedded Systems} \cite{yocto-or-debian} podemos matizar nuestra respuesta en consideración.

\begin{displayquote}
	``\textit{Debian} es bueno para pruebas rápidas y entornos de escritorio con memorias grandes y requisitos bajos de mantenimiento. \textit{Yocto} es bueno para entornos personalizados con soporte a distintas plataformas hardware de poca memoria y que requieren trazabilidad y reusabilidad.
\end{displayquote}

Sobre \textit{Debian}, también menciona que al ser conveniente para pruebas rápidas, muchos equipos de desarrollo comienzan su trabajo en la plataforma embebida utilizándolo, posponiendo tanto el paso a utilizar \textit{Yocto} para diseñar un sistema propio, que terminan postérgandolo infinitamente y no lo realizan nunca. Un argumento a favor de ellos es que la \textbf{curva de aprendizaje} de \textit{Yocto} es grande, dado que requiere compilar todo el sistema para la arquitectura destino, mientras que opciones como \textit{Debian} seguramente ya dispongan de paquetes precompilados.\\

Por ahora, nos conformaremos con saber de la existencia de estas alternativas. Más adelante comentaremos cuál nos resulta más conveniente para el caso que nos ocupa.\\


\subsubsection{Programación del robot}

Pasemos ahora a hablar de \textbf{software}. ¿Qué herramientas necesita un desarrollador para programar su robot? Pues bien, indagando un poco podremos ver que realmente no dista mucho de codificar cualquier otro software convencional.

TODO: describir sucíntamente ROS y por qué no es necesario usarlo existiendo Redis

\chapter{Especificación y requisitos}

Pasemos ahora a describir de la forma más detallada posible cada uno de los requerimientos que conforman la idea del proyecto, planteando del mismo modo alternativas o aspectos que sería de agrado incluir, si bien no forman parte de la especificación inicialmente. Las decisiones tomadas, así como las soluciones implementadas, serán detalladas en el capítulo siguiente, si bien a lo largo de éste mismo pueden surgir necesidades cuya solución se detalle directamente.\\

Se desea disponer de una plataforma robótica extensible, libre y abierta; que permita una buena ampliación de nuevas características mediante software. Además, se propone la implementación de un uso concreto para esta plataforma, y es que dicho robot sirva como \textbf{guía de caminos} para sus usuarios finales en un \textbf{entorno controlado}. En este capítulo listaremos los distintos requisitos y puntualizaremos sobre cada una de las decisiones tomadas para satisfacerlos.\\

\section{Requisitos generales}

Los requerimientos aquí listados comprenderán cosas tanto de procedimientos para el desempeño del proyecto (como pueden ser la visibilidad y su legislación) hasta las funcionalidades más concretas que se esperan del producto final.\\

\subsection{Licencias}

Se tratará de un proyecto de software \textbf{libre} y de \textbf{código abierto}. Para ello, se publicará bajo una licencia \textit{GPLv3} en un repositorio público en \textit{Github}.\\

Además de \textit{Github}, existen otras alternativas de repositorios públicos que permiten utilizar \textit{git} para control de versiones (como \textit{Bitbucket} y \textit{Gitlab} entre otras). El porqué de utilizar \textit{Github} es meramente por aprovechar la licencia \textit{premium} que se provee a los estudiantes de la UGR simplemente por matricularse; permitiendo así tener algunos repositorios privados a su disposición \cite{github-premium}.\\

Con el código público y la licencia elegida, cualquier usuario podrá descargar el software, compilarlo, ejecutarlo e incluso añadir modificaciones al código para ser probadas e integradas en la plataforma final. Para ello, en el propio repositorio se pondrá a disposición de los interesados una documentación que explique cómo replicar el entorno tanto de desarrollo como de compilación.\\


\subsection{Especificaciones}

El término \textit{plataforma robótica} puede ser demasiado amplio para su manejo, por lo que en esta sección detallaremos más a fondo algunos de los puntos más interesantes, así como los factores de éxito que harían de \textbf{\textit{Lazarillo}} un producto útil y diferencial con respecto a las alternativas ya existentes.\\


\subsubsection{Extensibilidad}

Ya que se desea disponer de una plataforma extensible cuyo comportamiento y características puedan ampliarse a través de software, se debe dotar al robot de una arquitectura que permita este crecimiento, conteniendo en ella servicios (o módulos) independientes que puedan incluirse o no en función de la aplicación específica que vaya a cumplir el robot.\\

Para ello, sería interesante disponer de una arquitectura basada en \textbf{microservicios} donde cada uno de ellos cumple un propósito muy concreto y se comunica con el resto sin generar acoplamiento. Para esto es de imperativa necesidad utilizar un paradigma de comunicación en el cual sea transparente añadir datos y servicios.\\

Por otro lado, ha de asegurarse que existe la \textbf{interoperabilidad}, la cual representa que una arquitectura que incluye distintas plataformas de hardware y diferentes sistemas software (implementados con lenguajes de programación variados) cooperan bien entre sí.\\

Veamos un ejemplo rápido para ilustrar el párrafo anterior: si el programa que recibe los datos de un servidor web está programado en \textit{Python} y debe transmitirlos al servicio que toma las decisiones de movimiento (programado en \textit{C++}); el paradigma de comunicación que los conecta ha de ser \textbf{agnóstico en el lenguaje} y que dicha operación sea efectiva de forma transparente.\\


\subsubsection{Conectividad}

Como plataforma inteligente y conectada que utiliza el paradigma del \textbf{\textit{edge computing}}, sabemos que el robot ha de ser un dispositivo embebido con conexiones al exterior como \textit{Bluetooth} y/o \textit{WiFi}. Que éstas vengan implícitas en la plataforma hardware utilizada facilitará mucho el trabajo, ya que nos ahorramos el hecho de tener que soldar componentes. En cuanto a plataformas hardware, en el apartado del \textbf{Estado del Arte} ya comentamos algunas alternativas y opciones. Utilizaremos para ésto una \textit{Raspberry Pi Model 3 B}, que pese a no ser el último modelo de la marca \textit{Raspberry}, es la que tengo a disposición en casa. Además, satisface las necesidades de conectividad que comentábamos.\\

Por otro lado, en cuanto al requisito de computación en el borde, la plataforma deberá contener uno o más servicios que permitan la comunicación con el exterior, de una forma u otra, además de enviar y recibir mensajes. El robot deberá habilitar un canal de comunicación \textbf{persistente} y \textbf{bidireccional} que le permitan tanto a él como al servidor enviarse eventos entre sí.\\


\subsubsection{Interfaz y experiencia de usuario}

El robot contará con una pantalla táctil con la que proveerá la información necesaria al usuario (en función de las aplicaciones que necesite o decidan integrarse en el robot). Cualquier pantalla táctil que permita su conexión con la \textit{Raspberry} debería servir, por lo que no entraremos a detallar limitaciones hardware. Para la interacción del usuario, el sistema contará con una \textbf{aplicación embebida de entorno gráfico} que permita el uso del robot.\\

Dado que \textbf{\textit{Lazarillo}} se pretende que actúe como \textbf{guía}, otra característica que sería de agradecer en la plataforma tiene que ver con la \textbf{reproducción de sonidos} que faciliten la comunicación con el usuario, así como la \textbf{accesibilidad}. No todo el mundo goza de capacidad visual o simplemente no están habituados a interfaces táctiles, por lo que emitir alertas y sonidos descriptivos facilitaría llegar a más usuarios de forma plena y satisfactoria.\\

\subsubsection{Gestión experta y mantenimiento}

Como hemos comentado en secciones anteriores, el robot gozará de hardware provisto de conectividad inalámbrica. En este punto haremos uso de esta característica para ofrecer un método de mantenimiento, supervisión y gestión del robot, por parte de alguna "mano experta". Necesitaremos un método de administración de los distintos robots existentes desde un portal web externo a ellos. Un técnico encargado de gestionar los robots, accederá a una web alojada en un servidor a través del protocolo común de \textit{HTTP}.\\

Inicialmente, este portal web servirá para listar los dispositivos conectados (es decir, los robots que han sido provisionados con el software de \textit{Lazarillo} y se encuentran en funcionamiento), pero posteriormente permitirá enviar acciones remotas desde el servidor al robot (como reinicios, actualizaciones de software, acciones concretas a realizar por el robot, etc.).\\

Es deseable que la interfaz web sea sencilla y usable. Además, sería de agradecer que ésta pueda visualizarse correctamente en \textbf{dispositivos móviles}, ya que ampliaría el rango de posibilidades de gestión de los dispositivos robóticos.\\

\subsubsection{Movilidad}

El factor determinante que diferenciará a nuestro robot de un sistema empotrado inmóvil será la capacidad de desplazarse por el entorno. Ya sea para un uso u otro, el robot deberá venir dotado de un sistema hardware que le permita avanzar por el plano, conteniendo elementos como \textbf{motores} y \textbf{ruedas} o \textbf{cintas móviles}.\\

Además, si se desea que el robot sea \textbf{inteligente} y reconozca el entorno por el que se está moviendo, deberá dotarse de algún sistema de reconocimiento como \textbf{sensores de proximidad} o \textbf{cámaras}. Respecto a esto, si queremos seguir el paradigma de \textit{edge computing} como venimos comentando, el procesamiento de estas señales podría realizarse en el servidor en lugar de en el propio robot, lo que también liberaría a la plataforma hardware del robot de una buena parte de la computación.\\

Para acotar el alcance del proyecto y que sea asumible para un trabajo de este calibre, inicialmente la movilidad podrá estar basada en hacer \textbf{seguimiento de líneas} sobre el suelo.\\

\begin{figure}[h]
	\centering
	\includegraphics[width=0.6\textwidth]{imagenes/robotnik.jpg}
	\caption{Ejemplo de robot móvil autónomo - Fuente: \textit{Robotnik}}
\end{figure}


\section{Objetivos opcionales}

En esta sección enumeraremos y describiremos sucíntamente qué otras ideas surgieron durante el momento de \textit{brainstorming} del proyecto, y que si bien son opcionales para su desempeño, realmente aportarían algún valor al producto final.\\


\subsubsection{Movilidad autónoma}

Un factor que haría de \textbf{\textit{Lazarillo}} un producto totalmente independiente y útil sería que no necesitase de caminos guiados para desplazarse. Se valoraría la instalación en sí mismo de los mapas cerrados en que se ubicaría durante su desempeño, así como un mecanismo de \textbf{geolocalización} en el espacio. Contando con esto, el robot mantendría una comunicación persistente con el servidor informando en \textit{tiempo real} de la ubicación actual.\\






\chapter{Implementación}

En este capítulo comentaremos a fondo las decisiones tomadas que hemos comentado previamente en base a la especificación, además de describir las implementaciones propuestas con las distintas tecnologías utilizadas. Pondremos el foco no solo en el \textbf{software implícito en el robot} sino también a los componentes desarrollados externos a él, pero que también conforman la arquitectura completa de la plataforma, como lo son el \textbf{sistema operativo}, el \textbf{panel web de administración} y otras herramientas involucradas. Las ideas que por una razón u otra no hayan llegado a implementarse a tiempo, serán definidas en profundidad en apartados posteriores.\\

Para elaborar este capítulo de una forma más legible y entendible, seguiremos una estructura muy similar a la propuesta en el capítulo 3 de \textbf{Especificación de requisitos}.\\


\section{Requisitos generales}


\subsection{Licencias y compartición del software}

En cuanto a licencias y permisos del proyecto, como se anticipó se usarán diversos \textbf{repositorios} en \textit{Github}, que aparecen listados a continuación y descritos en función del contenido (que se elaborará más tarde):

\begin{itemize}
	\item \textbf{\textit{lazarillo-embedded}}: \cite{lazarillo-embedded}
	\item \textbf{\textit{lazarillo-admin-frontend}}: \cite{lazarillo-admin-frontend}
	\item \textbf{\textit{lazarillo-admin-backend}}: \cite{lazarillo-admin-backend}
	\item \textbf{\textit{meta-lazarillo}}: \cite{meta-lazarillo}
	\item \textbf{\textit{documentacion-tfm}}: \cite{documentacion-tfm}
\end{itemize}


\section{Sistema operativo}

Para dotar a la \textit{Raspberry} de un sistema operativo ligero, extensible y hecho a medida, decidimos obviar sistemas ya existentes como \textit{Raspbian} (S.O. de culto para la mayoría de usuarios de este computador) (TODO: ref a Raspbian) y confeccionar el nuestro propio a través de una herramienta libre y gratuita como es \textit{The Yocto Project} (TODO: ref a yocto).

\subsubsection{¿Por qué omitimos un sistema operativo que ya existe?}

Pues bien, la respuesta es fácil. \textit{Raspbian} es un sistema operativo de uso general que permite utilizar la \textit{Raspberry} como cualquier ordenador normal, ya sea para ofimática, desarrollo de software, consumo de recursos multimedia o incluso videojuegos. Esto provoca que el sistema operativo en cuestión venga con demasiado \textit{bloatware} (TODO: poner en el glosario) preinstalado de base; y llevaría más tiempo modificar la imagen de \textit{Raspbian} para que no contenga todo el software no deseado que confeccionar un sistema operativo a medida.\\

Además, deseamos que el robot solo muestre una aplicación embebida en pantalla en lugar de un entorno de escritorio normal, por lo que podemos prescindir de este entorno completo y configurar nuestro nuevo sistema para que solo muestre una aplicación y ahorre tiempo en el arranque.\\


\section{Arquitectura de Lazarillo}

En esta sección enumeraremos los distintos servicios y módulos específicos creados para Lazarillo y describiremos el propósito para el que han sido programados.\\

(TODO: insertar diagrama de arquitectura)

\subsection{Servicios internos}

Veamos ahora una breve explicación del propósito que satisfacen los módulos software internos del robot.

\subsubsection{lazarillo-hmi}

La tecnología usada por esta aplicación, se ha decidido que sea el framework de \textbf{\textit{Qt}}, que pese a tener modalidades de licencias de pago, permite un uso \textit{open source}. Además, se trata de una tecnología con una amplia comunidad, una documentación muy rica y un gran soporte para dispositivos embebidos. Por otro lado, dispongo de amplia experiencia con dicha herramienta, lo que agiliza enormemente el tiempo de desarrollo.

Se trata de la aplicación que muestra la interfaz táctil al usuario.

\subsubsection{messages-definition}

Este módulo no es en sí un servicio sino una \textbf{librería} donde se definen los mensajes que utilizan internamente el resto de servicios para comunicarse.

\subsubsection{service-base}

Este módulo define el esqueleto abstracto de cada uno de los servicios del robot. Heredando el comportamiento de esta librería, cada uno de los nuevos servicios se ahorran procedimientos rutinarios como instanciar la conexión con la base de datos y el bróker de mensajería.

\subsubsection{web-gateway}

Este servicio es el \textit{puerto de entrada} a Lazarillo. Realiza la comunicación mediante \textit{Websockets} con \textbf{\textit{lazarillo-admin}}, el portal web a través del cual un técnico puede administrar los distintos robots. \textit{\textbf{Web-gateway}} actúa como intérprete de los mensajes provenientes del \textit{socket} y los publica en la mensajería interna del robot (basada en \textit{Redis}) para informar al resto de servicios de cualquier acción emitida por el servidor.


\subsection{Componentes externos}

Como venimos comentando, además del software embebido en el robot, disponemos de otros módulos externos a él que completan la plataforma.

\subsubsection{lazarillo-admin}

Este es el título asignado al servicio web que tanto venimos comentando a lo largo de este documento, a través del cual podemos visualizar y gestionar los distintos robots conectados al servidor.\\

Está programado con \textit{NodeJS} (TODO: ref a node) y utiliza librerías para hacer las veces de servidor con \textit{HTTPS} (para la web) y con \textit{Websockets} (para la comunicación con el robot).\\

\chapter{Trabajos futuros}

Un buen \textit{brainstorming} sobre el desarrollo de una idea novedosa y (se supone que) útil siempre es motivador y consigue animar a un programador a llevar la idea a la práctica, intentando la consecución del mayor número de requisitos posibles. Sin embargo, el tiempo siempre es uno de los factores más determinantes a la hora de dictaminar cómo de lejos se llega con el proyecto.\\

Dicho esto, en este capítulo comentaremos cuáles de las ideas iniciales no llegaron a finalizarse a tiempo para la entrega del proyecto, y cuáles se decidieron postergar en el tiempo para un futuro desarrollo.\\

Cabe destacar que dada la naturaleza libre del proyecto, cualquier interesado o interesada podría continuar con la parte del desarrollo que más le interese; ya que esta idea nació como una herramienta abierta a la que poder contribuir y de la que cualquiera pueda sacar utilidad.\\

Entrando ya en materia, dividiremos este capítulo en breves secciones relacionadas con las ya vistas en el capítulo de \textbf{Implementación}, detallando los cabos sueltos que hayan quedado en cada uno de ellos.

\section{Sistema operativo}

\section{Servicios propios}

\section{Otros}

TODO: rellenar secciones (tras terminar capítulo de Implementación)
\chapter{Conclusiones y trabajos futuros}

Un buen \textit{brainstorming} sobre el desarrollo de una idea novedosa y (se supone que) útil siempre es motivador y consigue animar a un programador a llevar la idea a la práctica, intentando la consecución del mayor número de requisitos posibles. Sin embargo, el tiempo siempre es uno de los factores más determinantes a la hora de dictaminar cómo de lejos se llega con el proyecto, por lo que siempre existe la posibilidad de que no se alcance a finalizar todas las ideas que rondan la cabeza.\\

Personalmente, tener que compaginar mi jornada completa laboral con el desempeño de este trabajo ha hecho mucha mella en lo que fue la primera idea del proyecto. Me habría gustado poder acercarme realmente más al hardware, disponer de un robot sencillito al que implementar unas directivas muy básicas de movimiento pero que realmente hiciesen tangible el trabajo desarrollado en ésto.\\

También me gustaría destacar que, después de años de experiencia profesional, realmente quería tomar buenas decisiones, plantear bien la arquitectura, usar las herramientas idóneas para cada caso, establecer buenas prácticas a largo plazo para asegurar que si yo no puedo hacerlo, alguien podría acceder a los repositorios y darle continuidad a estas finalidades que han quedado en el limbo.\\

Cabe destacar que dada la naturaleza libre del proyecto, cualquier interesado o interesada podría continuar con la parte del desarrollo que más le interese; ya que esta idea nació como una herramienta abierta a la que poder contribuir y de la que cualquiera pueda sacar utilidad.\\

Dicho esto, en este capítulo comentaremos algunas de las ideas iniciales que no llegaron a finalizarse a tiempo para la entrega y cuáles se decidieron postergar en el tiempo para un futuro desarrollo.\\

Para empezar, lo más ligado a lo que pretendía ser el epicentro del proyecto: la \textbf{robótica}. Sería idóneo instalar el sistema operativo preparado al efecto en un robot real cuyo núcleo de computación sea la \textit{Raspberry}. Tras esto, conectar algunos motores y unas ruedas para realizar un movimiento autónomo aunque fuese muy sencillo (implementando el módulo \textbf{\textit{motor-manager}} que se comentó en la arquitectura).\\

En cuanto al resto del código embebido en el cargador, queda pendiente la aplicación embebida mostrada en la pantalla táctil. En ella el usuario introducirá el destino al que quiere ir, el despacho del que desea saber cómo llegar, etc.\\

En cuanto a los servicios web, un aspecto importante que no llegó a gestionarse fue el de la autenticación del administrador. Obviamente para una gestión experta de dispositivos delicados, no cualquiera debería tener acceso a las herramientas, por lo que los desarrolladores del servicio web deberían crear cuentas autorizadas para que los administradores puedan autenticarse en el portal.\\

Hilando un poco más fino, los archivos \textit{Dockerfile} con los que se generan los contenedores incluyen variables de entorno en texto plano como usuarios y contraseñas de la base de datos. Lógicamente esto para un proyecto de prueba es aceptable temporalmente pero orientado a desplegarse en producción, lo ideal es leer los valores de variables del sistema local pero que no aparezcan en ningún fichero.\\

En cuanto a la conectividad, ya lo hemos ido viendo, pero el socket que conecta la web con el robot debería hacer comprobaciones periódicas de la lista de dispositivos existentes.\\


% \nocite{*}
\bibliographystyle{ieeetr}
\bibliography{bibliografia/bibliografia}

%\appendix
%\input{glosario/entradas_glosario}
% \addcontentsline{toc}{chapter}{Glosario}
% \printglossary

\end{document}
