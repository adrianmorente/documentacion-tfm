\chapter{Introducción}

\textbf{Lazarillo} se trata de una plataforma robótica abierta cuyo propósito es proporcionar una arquitectura solvente y extensible que permita añadir nuevas características y utilidades, además de facilitar el acceso a su gestión y mantenimiento.\\

Contiene un sistema operativo abierto basado en GNU/Linux y \textit{The Yocto Project}, además de distintos servicios implementados con lenguajes de programación diferentes, mostrando así la interoperabilidad del software. Goza de conectividad inalámbrica, la cual facilita la conexión con herramientas externas para su gestión. Asímismo, el proyecto también consta de un panel web de administración desde el cual se pueden enviar acciones remotas al robot.\\

Aunque se trata de una plataforma extensible y de propósito múltiple, el uso inicial para el que fue ideado es el de actuar como \textbf{asistente} y \textbf{guía} dentro de un espacio cerrado (ya podemos ver que el título asignado al proyecto es un pequeño guiño a la literatura española). Sin embargo, el \textit{stack} de herramientas y arquitectura que se han ido confeccionando durante su desarrollo, se podrían utilizar fácilmente para cualquier proyecto de propósito general que aúne dispositivos embebidos con \textit{IoT} y administración remota de éstos.\\

En primer lugar, analizaremos el \textit{Estado del arte} de todas las tecnologías y herramientas que puedan estar involucradas en un proyecto que inicialmente está orientado a la robótica.\\

Para continuar, veremos al detalle la especificación de requisitos tanto generales de arquitectura como específicos de funcionalidad. Acto seguido nos entretendremos comentando la implementación provista para cada uno de los módulos involucrados en el proyecto, argumentando cada una de las decisiones tomadas y el objetivo a largo plazo que persiguen.\\

Finalmente, hablaremos de posibles puertas que abre el estado actual del proyecto para otros trabajos futuros, sea para darle continuidad a esta idea inicial o bien para aprovechar toda la infraestructura y orientarla a otro proyecto que también pueda aprovecharla.\\

