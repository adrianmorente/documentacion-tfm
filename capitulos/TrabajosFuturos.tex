\chapter{Trabajos futuros}

Un buen \textit{brainstorming} sobre el desarrollo de una idea novedosa y (se supone que) útil siempre es motivador y consigue animar a un programador a llevar la idea a la práctica, intentando la consecución del mayor número de requisitos posibles. Sin embargo, el tiempo siempre es uno de los factores más determinantes a la hora de dictaminar cómo de lejos se llega con el proyecto.\\

Dicho esto, en este capítulo comentaremos cuáles de las ideas iniciales no llegaron a finalizarse a tiempo para la entrega del proyecto, y cuáles se decidieron postergar en el tiempo para un futuro desarrollo.\\

Cabe destacar que dada la naturaleza libre del proyecto, cualquier interesado o interesada podría continuar con la parte del desarrollo que más le interese; ya que esta idea nació como una herramienta abierta a la que poder contribuir y de la que cualquiera pueda sacar utilidad.\\

Entrando ya en materia, dividiremos este capítulo en breves secciones relacionadas con las ya vistas en el capítulo de \textbf{Implementación}, detallando los cabos sueltos que hayan quedado en cada uno de ellos.

\section{Sistema operativo}

\section{Servicios propios}

\section{Otros}

TODO: rellenar secciones (tras terminar capítulo de Implementación)